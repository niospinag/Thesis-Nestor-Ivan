\chapter{Introduction}

With the coming of the Industrial Revolution, human beings started to use new machines for transport, like vehicles powered by combustion engines. These efficient and versatile machines made more accessible and faster the transportation of people, animals, and commodities. However, more than 200 years have passed since constructing the first automobile powered by a vapor engine. Due to the success of this new machine, it has undergone many modifications and upgrades to increase its efficiency, speed, and security. Nevertheless, the most crucial change in those vehicles is their energy source. The first mobile vehicles were powered by carbon, making the use of wood or carbon necessary to power the engine. It could be a great engine, but the massive load and space needed to store this material make it inefficient and annoying in its use. Later, the combustion engine was created by Étienne Lenoir around 1860. It uses fossil fuels like gasoline, natural gas, or diesel as a power source and fossil oils for its lubrication and maintenance. However the electric vehicle was invented in the 19th century, but its battery autonomy and energy capacity were inefficient. Eventually, the capacity and efficiency of electric engines and batteries have grown, making them more accessible, efficient, and powerful battery electric vehicles than internal combustion ones. Finally, the electric vehicle is manufactured and marketed in mass production by companies like Tesla, Chevrolet, and BMW.
\\

The use of electric energy is versatile because it can be transported and manipulated without significant losses; in addition, its weight is insignificant compared to oil and carbon. Besides, it can be easily transformed into multiple other kinds of energies, such as cinematic, thermic, mechanic, and electromagnetic. Due to this facility, electric energy is the best way to get, keep and use in different applications. It is advantageous to use this kind of energy as a source of power in a vehicle, which can be equipped with many power tools that make it an electric machine with many features. One of the great qualities is the automation system, which allows it to autonomously make acceleration, braking, and steering decisions. It makes a technological leap in safety and efficiency. Self-driving is very important because it will make changes in the actual way of transportation. A car will not need a human driver to go from one point to another. It means that an artificial driver will make decisions instead of a human. As a result, the artificial driver will drive efficiently, safely, cooperatively, and smartly. In conclusion, humanity will enjoy revolutionary transportation with the benefits of a car but without the disadvantages of the unsafely and the brain load of making decisions independently.
\\

The coming of electric vehicles into the world makes it possible to develop, build, and use computer systems that let vehicles be driven autonomously without the supervision of a human being, thus making human transportation safe, efficient, and optimal. However, the advantage of 5G technology is that electric and autonomous vehicles can communicate with others to achieve an individual goal cooperatively. On the whole, self-driving currently requires that vehicles cooperate,  considering the presence of human drivers. The human driver performs unknown strategies that hinder interaction on the road and lead to unsafe or inappropriate maneuvers. It is expected that autonomous driving will not need a human supervisor for safe working in the future. Due to this reason, in this thesis, we model a human driver as a selfish and non-cooperative agent who seeks their benefit. Simplifying autonomous driving in a natural environment, we formulate an environment with autonomous agents that share their physical variables (velocity, acceleration, lane number, and position) but ignore the non-cooperative agent (human driver) strategy. We intend to solve this problem using a novel method based on Nash equilibrium, which efficiently moves the cooperative agent to a specific objective. It considers each of the different objectives and the different dynamics of each agent. We also use ``Mixed Integer" variables to change the behavior of each autonomous agent in the presence of non-autonomous vehicles and ensure safety on the road. Finally, we use a technique of control MPC (Model Predictive Control) to predict the states of each agent on the whole network at a specific time in the future and decide the optimal strategy.\\
% Automated Driving
\section{Automated Driving}

Automated driving, also known as a self-driving car, is the action of a vehicle or machine to move through an environment making the right decisions to achieve goal and objective \cite{Taeihagh_2018}, \cite{Thrun2010TowardRC}. Automated vehicles combine techniques and hardware tools to capture, process, and control signals in their systems. Commonly, hardware tools allow it to sense the  Environment, e.g., GPS, lidar, radar, sonar, odometry, 3d vision, and cameras, among others. Moreover, these vehicles can use multiple control hardware such as power systems, electric engines, and actuators to interact with the environment  \cite{8957499}, \cite{811692}.
\\

This technology has multiple applications in personal transportation, package delivery, Robo-taxis, or platoons of connected vehicles transporting loads. It is worth mentioning that the autonomous driving theory could be used in any other application that solves the problem of
driving through a place, avoiding collisions, and arriving somewhere. Some of those applications could be a waiter, a dispenser pills nurse, a coffee dispenser, a tool mechanic assistant,
among others \cite{8957499},\cite{peter}.


\section{Control Strategies}
Autonomous driving is a challenging task to solve. Multiple solutions have been developed, like artificial intelligence, adaptive control, machine learning algorithms, and non-linear control \cite{1t_network, 2t_centraliz}. As a result, new strategies make managing and controlling complex systems easier with better accuracy and robustness than years before. Nevertheless, no one has an optimal solution to this challenge \cite{506394}.
\\

The principal feature to consider to solve this challenge is the environment data. It could give a controller enough information to achieve the desired position. Due to this, a suitable control technique and a very well set of parameters are essential. The right strategy will depend on what strategies are being used and when it could be better than any other. In this thesis, we will explain some better control strategies for a specific job and how those are implemented in the specific problem of automated driving.


\subsection{Model Predictive Control}
One of the most popular control theories implemented in non-linear systems is Model Predictive Control (MPC). This strategy is a process control method used to optimize a cost function while satisfying a set of constraints. In other words, model predictive control appears to be an optimal solution to predict future states while applying constraints to achieve their primary objective. This optimization requires advanced algorithms that solve quadratic problems with the presence of constraints \cite{GARCIA1989335}.
\\

In particular, MPC minimizes the value of a function representing the control problem's characteristics and objectives. Some could be power consumption, efficiency, precision, and others. Furthermore, it is essential to obtain a dynamic model representing the entire system's behaviour. This model is the principal property needed for an MPC to do the control task efficiently. Depending on the conditions and properties of the problem control, it is also essential to implement some constraints that contain and restrain the entire system drop under incorrect conditions \cite{MAYNE2000789}.\\



% ----------------------- ADMM -------------------------
% \subsection{Alternating Direction Method of Multipliers}





\section{Game Theory}

Game theory is defined in \cite{33t_GameTheory2} as ``The study of mathematical models of conflict and cooperation between intelligent, rational decision-makers". One way to analyze the control game theory is in cooperative games, where Two or more players try to get the most benefit possible depending on neighbors' decisions. Each agent has to take into account the action of the others to make an optimal decision benefiting the whole group. 
The primary motivation of game theory is to achieve maximum profit to get close to or achieve the goal. Moreover, a game theory is implemented from a cooperative perspective. There can be many agents with the same objective function, and each cooperative agent must make the optimal decision to achieve the global objective. Game theory is functional in environments with multiple interacting controllers, and the decision of each one depends on the other.

